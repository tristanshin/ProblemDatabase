Fill in the following (currently blank) grid so that:
	\begin{itemize}
		\item Each box contains a single digit in the set $\{1,\dots,9\}$.
		\item A \vocab{segment} is the set of boxes in a row or column from one black box to another, with no black boxes in between. There are $16$ segments in this grid.
		
		If two boxes are in the same segment, then they contain different digits.
		\item The digits in each segment sum to the number labelling the segment.
	\end{itemize}
	\begin{center}
		\begin{asy}
			defaultpen(fontsize(15pt));
			size(15cm);
			pair[] blacksquares = {
				(0,0), (0,1), (0,2), (0,3), (0,4), (0,5), (0,6),
				(1,0), (1,1), (1,5), (1,6),
				(2,6),
				(3,2), (3,3), (3,6),
				(4,2), (4,3), (4,6),
				(5,6),
				(6,0), (6,4), (6,5), (6,6),
			};
			int[] lowerlabels = {
				0, 0, 0, 0, 0, 0, 0,
				0, 0, 22, 0,
				35,
				9, 0, 9,
				12, 0, 3,
				25,
				0, 17, 0, 0,
			};
			int[] rightlabels = {
				0, 0, 9, 15, 15, 0, 0,
				13, 27, 26, 0,
				0,
				0, 0, 0,
				14, 13, 0,
				0,
				0, 0, 0, 0,
			};
			for (int j = 0; j < blacksquares.length; ++j) {
				pair P = blacksquares[j];
				fill(P -- P + (1,0) -- P + (1,1) -- P + (0,1) -- cycle, gray(0.7));
				draw(P + (1,0) -- P + (0,1));
				if (lowerlabels[j] != 0) {
					label(string(lowerlabels[j]), P + (0.25, 0.25));
					draw(P + (0.5,0.35) -- P + (0.5,0.1), arrow = Arrow(TeXHead));
				}
				if (rightlabels[j] != 0) {
					label(string(rightlabels[j]), P + (0.75, 0.75));
					draw(P + (0.65,0.5) -- P + (0.9,0.5), arrow = Arrow(TeXHead));
				}
			}
			for (int i = 0; i <= 7; ++i) {
				draw((i,0) -- (i,7));
				draw((0,i) -- (7,i));
			}
		\end{asy}
	\end{center}
	