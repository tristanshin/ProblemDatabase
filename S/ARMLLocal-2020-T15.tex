Let the circles be $\Gamma_1$ and $\Gamma_2$ with radii $r_1$ and $r_2$. Let $\Gamma_1$, $\Gamma_2$, and the incircle of triangle $ABC$ have centers $I_1$, $I_2$, and $I$, respectively, and touch side $BC$ at $D$, $E$, and $X$, respectively. By definition, $DI_1=r_1$, $EI_2=r_2$, and $XI=r$. Then $BDI_1$ and $BXI$ are similar right triangles, so $\frac{BD}{r_1}=\frac{BX}{r}=\frac{s-b}{r}$ where $s$ is the semiperimeter. It follows that $BD=\frac{r_1(s-b)}{r}$ and similarly $CE=\frac{r_2(s-c)}{r}$.
	\begin{center}
		\begin{asy}
			size(8cm);
			real a=18, b=13, c=17, r1=3*sqrt(77)/11, r2=3*sqrt(77)/7, s=(a+b+c)/2, K=sqrt(s*(s-a)*(s-b)*(s-c)), r=K/s, h=2*K/a;
			pair A=(sqrt(c^2-h^2),h), B=(0,0), C=(a,0), I1=(a/3,r1), I2=(2*a/3,r2), I=(s-b,r), D=(a/3,0), E=(2*a/3,0), X=(s-b,0), O=(a/2,a*(b^2+c^2-a^2)/8/K);
			draw(A--B--C--cycle); draw(circle(I1,r1)); draw(circle(I2,r2)); draw(D--I1--I2--E); draw(B--I--C,dashed); draw(X--I,dashed);
			dot(A); dot(B); dot(C); dot(I1); dot(I2); dot(I); dot(D); dot(E); dot(X);
			label("$A$",A,dir(O--A)); label("$B$",B,dir(O--B)); label("$C$",C,dir(O--C)); label("$I_1$",I1,dir(B--I)*dir(90)); label("$I_2$",I2,dir(C--I)*dir(-90)); label("$I$",I,dir(90)); label("$D$",D,dir(-90)); label("$E$",E,dir(-90)); label("$X$",X,dir(-90));
		\end{asy}
	\end{center}
	From right trapezoid $DI_1I_2E$ it follows that $DE^2+(r_2-r_1)^2=(r_2+r_1)^2$ by the Pythagorean Theorem. Simplifying this gives $DE=2\sqrt{r_1r_2}$, so
	\[
		a=BD+DE+EC=\frac{r_1(s-b)}{r}+\frac{r_2(s-c)}{r}+2\sqrt{r_1r_2}.
	\]
	Applying AM-GM on $r_1(s-b)$ and $r_2(s-c)$ gives $\frac{1}{2}\left(r_1(s-b)+r_2(s-c)\right)\geq\sqrt{r_1r_2(s-b)(s-c)}$ so
	\[
		a\geq\frac{2\sqrt{r_1r_2(s-b)(s-c)}}{r}+2\sqrt{r_1r_2}=2\sqrt{r_1r_2}\left(\sqrt{\frac{s}{s-a}}+1\right)
	\]
	since $r=\frac{[ABC]}{s}=\frac{\sqrt{s(s-a)(s-b)(s-c)}}{s}$ by Heron's Formula.
	
	Applying AM-GM on $\frac{4(s-a)}{s}$, $\frac{1}{2}\sqrt{\frac{s}{s-a}}$, and $\frac{1}{2}\sqrt{\frac{s}{s-a}}$ gives $\frac{1}{3}\left(\frac{4(s-a)}{s}+\sqrt{\frac{s}{s-a}}\right)\geq1$ so
	\[
		a\geq2\sqrt{r_1r_2}\left(4-\frac{4(s-a)}{s}\right)=a\cdot\frac{8\sqrt{r_1r_2}}{s}.
	\]
	Since $r_1r_2=9$ and $s=24$, the right hand side is just $a$ so equality must hold in each inequality used. Thus $r_1(s-b)=r_2(s-c)$ and $\frac{4(s-a)}{s}=\frac{1}{2}\sqrt{\frac{s}{s-a}}$. The latter implies $\frac{s-a}{s}=\frac{1}{4}$, so $a=\frac{3s}{4}=18$. With $b=13$, it follows that $c=17$. Then $11r_1=7r_2$, but $r_1r_2=9$, so $r_1=3\sqrt{\frac{7}{11}}$ and $r_2=3\sqrt{\frac{11}{7}}$. So the answer is
	\[
		r_1+r_2=3\left(\sqrt{\frac{7}{11}}+\sqrt{\frac{11}{7}}\right)=\frac{3}{77}\left(7\sqrt{77}+11\sqrt{77}\right)=\mathbf{\frac{54\sqrt{77}}{77}}.
	\]