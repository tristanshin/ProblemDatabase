Let $W$ be the intersection of the $A$- and $D$-exterior angle bisectors, and $Z$ the intersection of the $B$- and $C$-exterior angle bisectors. Let $\omega$ touch $AD$ at $G$ and $BC$ at $H$.

The key claim is that $GEHF$ and $WXZY$ are homothetic. The sides are clearly parallel because, for example, $GE\perp AI$ and $WX\perp AI$. Note that $AIBX$ is cyclic because of opposite right angles. In addition, $X,I,Y$ are collinear on the interior angle bisector of the angle formed by lines $AD$ and $BC$. Thus
\[
	\measuredangle{GEF}=\measuredangle{DGF}=\frac{\pi}{2}-\measuredangle{IDA}=\measuredangle{ABI}=\measuredangle{AXI}=\measuredangle{WXY}
\]
so since $GE\parallel WX$ we have that $EF\parallel XY$. Similarly, $GH\parallel WZ$, so the diagonals are parallel too. So the quadrilaterals are homothetic as claimed.

Now, let the diagonals of $GEHF$ meet at $P$. The diagonals of $WXZY$ meet at $I$ since $X,I,Y$ and $W,I,Z$ are collinear from before. Thus the homothety implies that $XE$, $YF$, and $IP$ are concurrent. But it is well-known (by polars) that $I,P,O$ are collinear, so the conclusion follows.