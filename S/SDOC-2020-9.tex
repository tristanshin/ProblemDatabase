First, look at the second horizontal segment. If it contains any digit larger than $5$, then its sum is greater than $15$. So the digits in this segment are $\{1,2,3,4,5\}$. Next, look at the first vertical segment. If it does not contain $9$, then its maximum possible sum is $8+7+6<22$, contradiction. So it contains a $9$. The other two digits sum up to $13$, and one of them is at most $5$. So the other two digits are $5$ and $8$. Consequently the topmost digit in this segment is $5$. The bottommost digit cannot be $9$ (because it is in a segment labelled by $9$), so the second digit is $9$, while the bottommost digit is $8$. This easily allows us to complete two horizontal segments also.
	\begin{center}
		\begin{asy}
			defaultpen(fontsize(8pt));
			size(7cm);
			pair[] blacksquares = {
				(0,0), (0,1), (0,2), (0,3), (0,4), (0,5), (0,6),
				(1,0), (1,1), (1,5), (1,6),
				(2,6),
				(3,2), (3,3), (3,6),
				(4,2), (4,3), (4,6),
				(5,6),
				(6,0), (6,4), (6,5), (6,6),
			};
			int[] lowerlabels = {
				0, 0, 0, 0, 0, 0, 0,
				0, 0, 22, 0,
				35,
				9, 0, 9,
				12, 0, 3,
				25,
				0, 17, 0, 0,
			};
			int[] rightlabels = {
				0, 0, 9, 15, 15, 0, 0,
				13, 27, 26, 0,
				0,
				0, 0, 0,
				14, 13, 0,
				0,
				0, 0, 0, 0,
			};
			for (int j = 0; j < blacksquares.length; ++j) {
				pair P = blacksquares[j];
				fill(P -- P + (1,0) -- P + (1,1) -- P + (0,1) -- cycle, gray(0.7));
				draw(P + (1,0) -- P + (0,1));
				if (lowerlabels[j] != 0) {
					label(string(lowerlabels[j]), P + (0.25, 0.25));
					draw(P + (0.5,0.35) -- P + (0.5,0.1), arrow = Arrow(TeXHead));
				}
				if (rightlabels[j] != 0) {
					label(string(rightlabels[j]), P + (0.75, 0.75));
					draw(P + (0.65,0.5) -- P + (0.9,0.5), arrow = Arrow(TeXHead));
				}
			}
			pair[] solutionsquares = {
				(1,2), (1,3), (1,4),
				(2,2), (2,3),
			};
			int[] solutionlabels = {
				8, 9, 5,
				1, 6,
			};
			for (int j = 0; j < solutionsquares.length; ++j) {
				pair P = solutionsquares[j];
				label(string(solutionlabels[j]), P + (0.5, 0.5), p = fontsize(12pt));
			}
			for (int i = 0; i <= 7; ++i) {
				draw((i,0) -- (i,7));
				draw((0,i) -- (7,i));
			}
		\end{asy}
	\end{center}
	Now, we turn our attention to the second vertical segment (labelled by $35$). We require four digits in $\{2,3,4,5,7,8,9\}$ to sum to $28$. If any of $9,8,7$ is excluded, then the maximum sum is $9+8+5+4=26$, contradiction. So all three of $9,8,7$ are included, resulting in $4$ being the last digit. Thus the remaining four digits in this segment are $\{4,7,8,9\}$. The only one that is at most $5$ is $4$, so $4$ must go in the intersection with the second horizontal segment. Now, if $8$ or $9$ is placed in the bottommost digit, then the bottommost horizontal segment requires three digits summing to $5$ or $4$. But the minimum possible value of the sum of three digits is $1+2+3=6$, contradiction. So $7$ must be placed in the bottommost digit.
	\begin{center}
		\begin{asy}
			defaultpen(fontsize(8pt));
			size(7cm);
			pair[] blacksquares = {
				(0,0), (0,1), (0,2), (0,3), (0,4), (0,5), (0,6),
				(1,0), (1,1), (1,5), (1,6),
				(2,6),
				(3,2), (3,3), (3,6),
				(4,2), (4,3), (4,6),
				(5,6),
				(6,0), (6,4), (6,5), (6,6),
			};
			int[] lowerlabels = {
				0, 0, 0, 0, 0, 0, 0,
				0, 0, 22, 0,
				35,
				9, 0, 9,
				12, 0, 3,
				25,
				0, 17, 0, 0,
			};
			int[] rightlabels = {
				0, 0, 9, 15, 15, 0, 0,
				13, 27, 26, 0,
				0,
				0, 0, 0,
				14, 13, 0,
				0,
				0, 0, 0, 0,
			};
			for (int j = 0; j < blacksquares.length; ++j) {
				pair P = blacksquares[j];
				fill(P -- P + (1,0) -- P + (1,1) -- P + (0,1) -- cycle, gray(0.7));
				draw(P + (1,0) -- P + (0,1));
				if (lowerlabels[j] != 0) {
					label(string(lowerlabels[j]), P + (0.25, 0.25));
					draw(P + (0.5,0.35) -- P + (0.5,0.1), arrow = Arrow(TeXHead));
				}
				if (rightlabels[j] != 0) {
					label(string(rightlabels[j]), P + (0.75, 0.75));
					draw(P + (0.65,0.5) -- P + (0.9,0.5), arrow = Arrow(TeXHead));
				}
			}
			pair[] solutionsquares = {
				(1,2), (1,3), (1,4),
				(2,0), (2,2), (2,3), (2,4),
			};
			int[] solutionlabels = {
				8, 9, 5,
				7, 1, 6, 4,
			};
			for (int j = 0; j < solutionsquares.length; ++j) {
				pair P = solutionsquares[j];
				label(string(solutionlabels[j]), P + (0.5, 0.5), p = fontsize(12pt));
			}
			for (int i = 0; i <= 7; ++i) {
				draw((i,0) -- (i,7));
				draw((0,i) -- (7,i));
			}
		\end{asy}
	\end{center}
	Now, look at the bottommost horizontal segment (labelled by $13$). From the earlier logic, the three remaining digits must be $\{1,2,3\}$. The middle among these three must be $3$, because that digit is in a segment labelled by $12$. Then the digit immediately above must be a $9$. Returning to the second vertical segment, we see that the bottommost remaining digit cannot be a $9$, so it must be an $8$. That allows us to finish this second vertical segment.
	\begin{center}
		\begin{asy}
			defaultpen(fontsize(8pt));
			size(7cm);
			pair[] blacksquares = {
				(0,0), (0,1), (0,2), (0,3), (0,4), (0,5), (0,6),
				(1,0), (1,1), (1,5), (1,6),
				(2,6),
				(3,2), (3,3), (3,6),
				(4,2), (4,3), (4,6),
				(5,6),
				(6,0), (6,4), (6,5), (6,6),
			};
			int[] lowerlabels = {
				0, 0, 0, 0, 0, 0, 0,
				0, 0, 22, 0,
				35,
				9, 0, 9,
				12, 0, 3,
				25,
				0, 17, 0, 0,
			};
			int[] rightlabels = {
				0, 0, 9, 15, 15, 0, 0,
				13, 27, 26, 0,
				0,
				0, 0, 0,
				14, 13, 0,
				0,
				0, 0, 0, 0,
			};
			for (int j = 0; j < blacksquares.length; ++j) {
				pair P = blacksquares[j];
				fill(P -- P + (1,0) -- P + (1,1) -- P + (0,1) -- cycle, gray(0.7));
				draw(P + (1,0) -- P + (0,1));
				if (lowerlabels[j] != 0) {
					label(string(lowerlabels[j]), P + (0.25, 0.25));
					draw(P + (0.5,0.35) -- P + (0.5,0.1), arrow = Arrow(TeXHead));
				}
				if (rightlabels[j] != 0) {
					label(string(rightlabels[j]), P + (0.75, 0.75));
					draw(P + (0.65,0.5) -- P + (0.9,0.5), arrow = Arrow(TeXHead));
				}
			}
			pair[] solutionsquares = {
				(1,2), (1,3), (1,4),
				(2,0), (2,1), (2,2), (2,3), (2,4), (2,5),
				(4,0), (4,1),
			};
			int[] solutionlabels = {
				8, 9, 5,
				7, 8, 1, 6, 4, 9,
				3, 9,
			};
			for (int j = 0; j < solutionsquares.length; ++j) {
				pair P = solutionsquares[j];
				label(string(solutionlabels[j]), P + (0.5, 0.5), p = fontsize(12pt));
			}
			for (int i = 0; i <= 7; ++i) {
				draw((i,0) -- (i,7));
				draw((0,i) -- (7,i));
			}
		\end{asy}
	\end{center}
	But then the $8$ in the second-to-bottommost horizontal segment prevents any other $8$s in this segment, so the first remaining digit in the bottommost horizontal segment must be $2$ and the digit above it must be $7$. This produces a $1$ in the last digit of the bottommost horizontal segment. Now, the second-to-bottommost horizontal segment requires two digits which sum to $3$, so these digits must be $\{1,2\}$. The $1$ cannot go first because there is a $1$ beneath it, so this row is determined.
	\begin{center}
		\begin{asy}
			defaultpen(fontsize(8pt));
			size(7cm);
			pair[] blacksquares = {
				(0,0), (0,1), (0,2), (0,3), (0,4), (0,5), (0,6),
				(1,0), (1,1), (1,5), (1,6),
				(2,6),
				(3,2), (3,3), (3,6),
				(4,2), (4,3), (4,6),
				(5,6),
				(6,0), (6,4), (6,5), (6,6),
			};
			int[] lowerlabels = {
				0, 0, 0, 0, 0, 0, 0,
				0, 0, 22, 0,
				35,
				9, 0, 9,
				12, 0, 3,
				25,
				0, 17, 0, 0,
			};
			int[] rightlabels = {
				0, 0, 9, 15, 15, 0, 0,
				13, 27, 26, 0,
				0,
				0, 0, 0,
				14, 13, 0,
				0,
				0, 0, 0, 0,
			};
			for (int j = 0; j < blacksquares.length; ++j) {
				pair P = blacksquares[j];
				fill(P -- P + (1,0) -- P + (1,1) -- P + (0,1) -- cycle, gray(0.7));
				draw(P + (1,0) -- P + (0,1));
				if (lowerlabels[j] != 0) {
					label(string(lowerlabels[j]), P + (0.25, 0.25));
					draw(P + (0.5,0.35) -- P + (0.5,0.1), arrow = Arrow(TeXHead));
				}
				if (rightlabels[j] != 0) {
					label(string(rightlabels[j]), P + (0.75, 0.75));
					draw(P + (0.65,0.5) -- P + (0.9,0.5), arrow = Arrow(TeXHead));
				}
			}
			pair[] solutionsquares = {
				(1,2), (1,3), (1,4),
				(2,0), (2,1), (2,2), (2,3), (2,4), (2,5),
				(3,0), (3,1),
				(4,0), (4,1),
				(5,0), (5,1),
				(6,1),
			};
			int[] solutionlabels = {
				8, 9, 5,
				7, 8, 1, 6, 4, 9,
				2, 7,
				3, 9,
				1, 2,
				1,
			};
			for (int j = 0; j < solutionsquares.length; ++j) {
				pair P = solutionsquares[j];
				label(string(solutionlabels[j]), P + (0.5, 0.5), p = fontsize(12pt));
			}
			for (int i = 0; i <= 7; ++i) {
				draw((i,0) -- (i,7));
				draw((0,i) -- (7,i));
			}
		\end{asy}
	\end{center}
	Now, look at the rightmost vertical segment. We require two digits to sum to $16$, so these digits must be $\{7,9\}$. But a $7$ cannot go in the horizontal segment labelled by $14$, so the digits must be $7$ and $9$ in that order. This allows us to place a $6$ and a $5$ immediately to the left. In addition, the second-to-rightmost vertical segment can also be completed. Indeed, the second digit in this segment is in $\{1,2,3\}$ but cannot be $1$ or $2$, so it must be $3$. Then the first digit is $8$.
	\begin{center}
		\begin{asy}
			defaultpen(fontsize(8pt));
			size(7cm);
			pair[] blacksquares = {
				(0,0), (0,1), (0,2), (0,3), (0,4), (0,5), (0,6),
				(1,0), (1,1), (1,5), (1,6),
				(2,6),
				(3,2), (3,3), (3,6),
				(4,2), (4,3), (4,6),
				(5,6),
				(6,0), (6,4), (6,5), (6,6),
			};
			int[] lowerlabels = {
				0, 0, 0, 0, 0, 0, 0,
				0, 0, 22, 0,
				35,
				9, 0, 9,
				12, 0, 3,
				25,
				0, 17, 0, 0,
			};
			int[] rightlabels = {
				0, 0, 9, 15, 15, 0, 0,
				13, 27, 26, 0,
				0,
				0, 0, 0,
				14, 13, 0,
				0,
				0, 0, 0, 0,
			};
			for (int j = 0; j < blacksquares.length; ++j) {
				pair P = blacksquares[j];
				fill(P -- P + (1,0) -- P + (1,1) -- P + (0,1) -- cycle, gray(0.7));
				draw(P + (1,0) -- P + (0,1));
				if (lowerlabels[j] != 0) {
					label(string(lowerlabels[j]), P + (0.25, 0.25));
					draw(P + (0.5,0.35) -- P + (0.5,0.1), arrow = Arrow(TeXHead));
				}
				if (rightlabels[j] != 0) {
					label(string(rightlabels[j]), P + (0.75, 0.75));
					draw(P + (0.65,0.5) -- P + (0.9,0.5), arrow = Arrow(TeXHead));
				}
			}
			pair[] solutionsquares = {
				(1,2), (1,3), (1,4),
				(2,0), (2,1), (2,2), (2,3), (2,4), (2,5),
				(3,0), (3,1),
				(4,0), (4,1),
				(5,0), (5,1), (5,2), (5,3), (5,4), (5,5),
				(6,1), (6,2), (6,3),
			};
			int[] solutionlabels = {
				8, 9, 5,
				7, 8, 1, 6, 4, 9,
				2, 7,
				3, 9,
				1, 2, 5, 6, 3, 8,
				1, 9, 7,
			};
			for (int j = 0; j < solutionsquares.length; ++j) {
				pair P = solutionsquares[j];
				label(string(solutionlabels[j]), P + (0.5, 0.5), p = fontsize(12pt));
			}
			for (int i = 0; i <= 7; ++i) {
				draw((i,0) -- (i,7));
				draw((0,i) -- (7,i));
			}
		\end{asy}
	\end{center}
	Finally, the second remaining digit in the first horizontal segment cannot be $1$, otherwise the first remaining digit in this segment must be $8$. So the second remaining digit is $2$, causing the first remaining digit to be $1$, which allows us to finish the board.
	