We proceed using complex numbers with the incircle as the unit circle, where lowercase letters denote the usual. Let the intouch triangle be at $1,\omega,\omega^2$ where $\omega=e^{i\cdot\frac{2\pi}{3}}$ and $1\in AB,\omega\in BC,\omega^2\in CA$. Pick $\gamma$ on the unit circle with argument between $\frac{2\pi}{3}$ and $\frac{4\pi}{3}$ as the tangency point of the variable line. Then by standard formulas, $a=\frac{2}{1+\omega}$, $b=\frac{2}{1+\omega^2}$, $p=\frac{2\omega\gamma}{\omega+\gamma}$, and $q=\frac{2\omega^2\gamma}{\omega^2+\gamma}$. Then we require $r$ to satisfy $|r-p|=|a-p|$ and $|r-q|=|b-q|$, so
\begin{align*}
	r\overline{r}-\overline{p}r-p\overline{r}+p\overline{p} &= a\overline{a}-\overline{p}a-p\overline{a}+p\overline{p} \\
	r\overline{r}-\overline{q}r-q\overline{r}+q\overline{q} &= b\overline{b}-\overline{q}b-q\overline{b}+q\overline{q}.
\end{align*}
One can check that $a\overline{a}=b\overline{b}=4$,
\[
	\overline{p}a+p\overline{a}=\frac{4}{(\omega+\gamma)(1+\omega)}+\frac{4\omega^2\gamma}{(\omega+\gamma)(1+\omega)}=\frac{4(\omega+\gamma)}{\omega(\omega+\gamma)(1+\omega)}=-4,
\]
and similarly $\overline{q}b+q\overline{b}=-4$, so we require
\begin{align*}
	r\overline{r}-\overline{p}r-p\overline{r} &= 8 \\
	r\overline{r}-\overline{q}r-q\overline{r} &= 8.
\end{align*}
Since $R$ is one of two intersections of two circles, there are two possible values of $r$. I claim that these values of $4\gamma$ and $-2\gamma$. Check that if $r=k\gamma$ for $k\in\{4,-2\}$,
\begin{align*}
	r\overline{r}-\overline{p}r-p\overline{r} &= k^2-\frac{2k\gamma}{\omega+\gamma}-\frac{2k\omega}{\omega+\gamma} = k^2-2k = 8 \\
	r\overline{r}-\overline{q}r-q\overline{r} &= k^2-\frac{2k\gamma}{\omega^2+\gamma}-\frac{2k\omega^2}{\omega^2+\gamma} = k^2-2k = 8
\end{align*}
so these are indeed the possible values of $r$.

Consequently, the possible locations of $R$ as $\ell$ varies are two $\frac{2\pi}{3}$ radian arcs, namely minor arc $AB$ of $(ABC)$ and the image of this arc under scaling by a factor of $-2$ at the center of $\triangle{ABC}$.
