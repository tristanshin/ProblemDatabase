For notational purposes, let $S+S$ be the set of positive integers that can be written as the sum of two elements of $A$. Then $|S+S|=D(S)\leq18$.

The key to the problem is to consider the function $r(x)$, defined to be the number of ways to write $x$ as an ordered sum of two elements of $S$. Then
\begin{align*}
	\sum_{x\in S+S} r(x) &= |S|^2=36 \\
	\sum_{x\in S+S} r(x)^2 &= E(S).
\end{align*}
It follows that
\[
	E(S) = \sum_{x\in S+S} r(x)^2 \stackrel{\text{Cauchy}}{\geq} \frac{1}{|S+S|}\left(\sum_{x\in S+S} r(x)\right)^2=\frac{|S|^4}{D(S)}\geq72.
\]
Now, I claim that $E(S)\equiv|S|\pmod4$. Let $B$ be the set of integers $2s$ for $s\in S$, and $C$ be the remaining elements in $S+S$ that are not in $B$. Note that if $s$ and $t$ are distinct elements of $S$, then $r(s+t)$ counts both $s+t$ and $t+s$. So:
\begin{itemize}
	\item If $x\in B$, then $r(x)$ is odd, so $r(x)^2\equiv1\pmod8$.
	\item If $x\in C$, then $r(x)$ is even, so $r(x)^2\equiv0\pmod4$.
\end{itemize}
Thus
\[
	E(S) = \sum_{x\in B} r(x)^2 + \sum_{x\in C} r(x)^2 \equiv \sum_{x\in B} 1 + \sum_{x\in C} 0 \equiv |B| \equiv |S| \pmod4.
\]
It follows that $E(S)\geq74$.

Suppose $E(S)=74$. From the above inequality, we have that $D(S)\geq\frac{|S|^4}{E(S)}>17$ so $D(S)=18$. We will derive a contradiction by attempting to represent $74$ as a sum of squares in the desired formulation. Observe that
\begin{align*}
	5^2+\underbrace{2^2+\dots+2^2}_{\text{$11$ times}}+\underbrace{1^2+\dots+1^2}_{\text{$6$ times}} &= 75 > 74 \\
	4^2+4^2+\underbrace{2^2+\dots+2^2}_{\text{$10$ times}}+\underbrace{1^2+\dots+1^2}_{\text{$6$ times}} &= 78 > 74
\end{align*}
so $r(x)<5$ for all $x$, and there is at most one value of $x$ for which $r(x)\geq4$.

Suppose that $r(x)=2$ for all $x\in C$. Then
\[
	\sum_{x\in B} r(x)^2 = 74-(\underbrace{2^2+\dots+2^2}_{\text{$12$ times}})=26.
\]
But the LHS is $|S|\equiv6\pmod8$, so this would be a contradiction. Thus there is an $x\in C$ for which $r(x)>4$. Combining these, we deduce that $r(x)=4$ for one $x\in C$ and $r(x)=2$ for the other $11$. It follows that
\[
	\sum_{x\in B} r(x)^2 = 74-(4^2+\underbrace{2^2+\dots+2^2}_{\text{$11$ times}})=14.
\]
It is clear that we require $r(x)=3$ for one $x\in B$ and $r(x)=1$ for the other $5$. But then
\[
	\sum_{x\in B} r(x) + \sum_{x\in C} r(x) = 3+\underbrace{1+\dots+1}_{\text{$5$ times}}+4+\underbrace{2+\dots+2}_{\text{$11$ times}}=34,
\]
contradiction. Thus $E(S)\neq74$. So $E(S)\geq78$ as desired.

Now, we show that $S=\{1,2,3,5,8,16\}$ gives $E(S)=78$. Check that $S+S=\{2,4,6,10,16,32\}\cup\{3,5,7,8,9,11,13,17,18,19,21,24\}$ so $D(S)=18$. To compute $E(S)$, check that for $s<t$ in $S$, $s+t$ is unique. Then check that $2+2=1+3$, $3+3=1+5$, and $5+5=2+8$, but no other elements of $B$ have other representations. It follows that
\[
	E(S) = 3^2+3^2+3^2+1^2+1^2+1^2+\underbrace{2^2+\dots+2^2}_{\text{$12$ times}} = 78
\]
as desired.