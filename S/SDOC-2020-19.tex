The primes that work are $p=3$ and $p\equiv\pm1\pmod{18}$, leading to the answer of $3+17+19+37+53+71+73+89=362$.
	
	First, check that $p=2$ does not work while $p=3$ does. Henceforth assume that $p>3$. The condition is equivalent to $X^6-X^3+1$ having roots in $\FF_p[\sqrt{-3}]$.
	
	The key claim is the following generalization of Euler's criterion: Let $q\equiv1\pmod n$ be a prime power. Then $a\in\FF_q$ is an $n$th power if and only if $a^{\frac{q-1}{n}}=1$ in $\FF_q$. To prove this, observe that $a$ is an $n$th power if and only if $X^n-a$ divides $X^q-X$. But $X^n-a$ divides $X^{q-1}-a^{\frac{q-1}{n}}$, so the claim is true.
	
	Let $q$ be the smallest power of $p$ for which we can embed $\sqrt{-3}$ in $\FF_q$. Then we require $\frac{1+\sqrt{-3}}{2}$ to be a cubic residue in $\FF_q$. By the claim, this requires
	\[
		\left(\frac{1+\sqrt{-3}}{2}\right)^{\frac{q-1}{3}}=1
	\]
	in $\FF_q$. If $\frac{q-1}{3}\not\equiv0\pmod6$, then it can be checked that this implies that $p=2$ or $3$. So $q\equiv1\pmod{18}$. All steps can be reversed, so the solutions are $q\equiv1\pmod{18}$.
	
	Now, we can see that this gives $p\equiv\pm1\pmod{18}$. Indeed, by quadratic reciprocity,
	\[
		\left(\frac{-3}{p}\right)=\left(\frac{-1}{p}\right)\left(\frac{3}{p}\right)=(-1)^{\frac{p-1}{2}}\left(\frac{p}{3}\right)(-1)^{\frac{3-1}{2}\cdot\frac{p-1}{2}}=\left(\frac{p}{3}\right)
	\]
	so:
	\begin{itemize}
		\item If $p\equiv1\pmod3$, then $\left(\frac{-3}{p}\right)=1$ so $q=p$ and we get $p\equiv1\pmod{18}$.
		\item If $p\equiv2\pmod3$, then $\left(\frac{-3}{p}\right)=-1$ so $q=p^2$ and we get $p\equiv-1\pmod{18}$.
	\end{itemize}
	So the answers are $\pm1\pmod{18}$ as claimed.
	