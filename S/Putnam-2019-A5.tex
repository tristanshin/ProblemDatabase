The answer is $\boxed{\frac{p-1}{2}}$.

The only number theoretic fact necessary is that the $m$th moment of $\FF_p$ for $m\not\equiv0\pmod{p-1}$ is $0$. There are a few different reasons this is true: (number theory) plug in a primitive root and use geometric series, (group theory) the sum is a multiple of the sum of a subgroup with zero sum, (algebra) use Newton sums on the polynomial $x^{p-1}-1$.

An immediate corollary is that for any polynomial that has $0$ as a root has degree $\leq p-2$, the sum of the polynomial over $\FF_p$ is $0$.

Take the $j$th derivative; we get
\[
	q^{(j)}(1)=\sum_{a\in\FF_p}a^{\frac{p-1}{2}}a^{\underline{j}}
\]
where $x^{\underline{j}}=x(x-1)\cdots(x-j+1)$ is the falling factorial. When $j=0,\ldots,\frac{p-3}{2}$, this is $0$ by the corollary. When $j=\frac{p-1}{2}$, the corollary implies that
\[
	q^{(j)}(1)=\sum_{a\in\FF_p}a^{p-1}=-1.
\]
So the first $\frac{p-1}{2}-1$ derivatives of $q$ at $1$ (a root of $q$) are $0$, so $1$ has multiplicity $\frac{p-1}{2}$ in $q$.