Let $r$ be the radius of the circle and define $D,E$ to be the centers of the circles as drawn.

\begin{center}
\begin{asy}
size(5cm);
real r=0.88999105559,s=1.44003577763;
pair B=(0,0), C=(5,0), A=(2,4), D=(s,r), E=(s+2r,r);
draw(A--B--C--cycle);
draw(circle(D,r));
draw(circle(E,r));
label("$A$",A,N);
label("$B$",B,SW);
label("$C$",C,SE);
label("$D$",D,SE);
label("$E$",E,SW);
dot(D);
dot(E);
\end{asy}
\end{center}

Now, draw in the trapezoid and incenter $I$ (as well as the foot of $I$ to $BC$, which we define as $K$) as shown:

\begin{center}
\begin{asy}
size(5cm);
real r=0.88999105559,s=1.44003577763;
pair B=(0,0), C=(5,0), A=(2,4), D=(s,r), E=(s+2r,r), I=(sqrt(5),(5-sqrt(5))/2), K=(sqrt(5),0);
draw(A--B--C--cycle);
draw(circle(D,r));
draw(circle(E,r));
draw(B--D--E--C,dashed);
draw(D--I--E,dashed);
label("$A$",A,N);
label("$B$",B,SW);
label("$C$",C,SE);
label("$D$",D,SE);
label("$E$",E,SW);
label("$I$",I,N);
label("$K$",K,S);
dot(D);
dot(E);
dot(I);
dot(K);
\end{asy}
\end{center}

Since $IK=4$ (inradius) and $BK=6$ (incircle properties), we can see from similar triangles that the tangent from $B$ to the left circle has length $\frac{3}{2}r$. Similarly, we have $CK=8$, so the tangent from $C$ to the right circle has length $2r$. We also know that the tangent between the two circles has length $2r$, so we can add along segment $BC$ to deduce that \[\frac{3}{2}r+2r+2r=14.\] Solving this gives that $r=\boxed{\frac{28}{11}}$.

Alternate solution: Let $r$ be the radius of the circle and define $D,E$ to be the centers of the circles as drawn.

\begin{center}
\begin{asy}
size(5cm);
real r=0.88999105559,s=1.44003577763;
pair B=(0,0), C=(5,0), A=(2,4), D=(s,r), E=(s+2r,r);
draw(A--B--C--cycle);
draw(circle(D,r));
draw(circle(E,r));
label("$A$",A,N);
label("$B$",B,SW);
label("$C$",C,SE);
label("$D$",D,SE);
label("$E$",E,SW);
dot(D);
dot(E);
\end{asy}
\end{center}

Now, draw in the three smaller triangles and the trapezoid as shown:

\begin{center}
\begin{asy}
size(5cm);
real r=0.88999105559,s=1.44003577763;
pair B=(0,0), C=(5,0), A=(2,4), D=(s,r), E=(s+2r,r);
draw(A--B--C--cycle);
draw(circle(D,r));
draw(circle(E,r));
draw(B--D--A--E--C,dashed);
draw(D--E,dashed);
label("$A$",A,N);
label("$B$",B,SW);
label("$C$",C,SE);
label("$D$",D,SE);
label("$E$",E,SW);
dot(D);
dot(E);
\end{asy}
\end{center}

We can compute the area of each of these in terms of $r$. First, look at $\triangle{ADB}$. By the standard area calculation, we get that this is $\frac{13}{2}r$. Similarly, the area of $\triangle{AEC}$ is $\frac{15}{2}r$.

Now, turn your attention to $\triangle{ADE}$. Note that the height of this triangle (with base $DE$) is $r$ less than the height of $\triangle{ABC}$ from $A$ by drawing a straight line in. We can calculate the height of $\triangle{ABC}$ from $A$ to be $12$, so the area of $\triangle{ADE}$ is $r\left(12-r\right)$ (since $DE=2r$). Now, trapezoid $BDEC$ has bases $BC=14$ and $DE=2r$ and height $r$, so thus has area $r\left(r+7\right)$. We can add these all up to get the area of $\triangle{ABC}$ (which is $84$), so \[r\left(r+7\right)+r\left(12-r\right)+\frac{13}{2}r+\frac{15}{2}r=84.\] Solving this for $r$ gives $r=\boxed{\frac{28}{11}}$.